\chapter{À propos}

\section*{Un petit mot du créateur}

J'espère que ce livret et ce moment vont vous plaire !
Je me suis bien amusé à le concevoir, c'est déjà ça...

Si un problème survient, ou que vous bloquez sur une énigme, que vous pensez avoir besoin d'aide\footnote{Par exemple si vous pensez qu'une erreur s'est glissée dans ce document…}, n'hésitez pas à me contacter par téléphone ou m'appeler
\input{telephones.tex}.

J'ai hâte de vous retrouver à la sortie afin de décerner le \emph{prix de la meilleure équipe !}
Toutes les équipes auront une petite récompense, ne vous inquiétez pas…


\section*{Aspect ``technique'' \& remarques geek}
Ce document a été rédigé et compilé par mes soins, en sélectionnant \emph{aléatoirement} les \nbenigmes{} énigmes parmi une liste contenant \totalnbenigmes{} énigmes.
%
Chaque énigme a été rédigée comme un petit document Markdown\footnote{\url{DaringFireball.net/projects/markdown/}},
qui est ensuite compilé en \LaTeX{} par \texttt{pandoc}\footnote{\url{pandoc.org/}}.
%
Le document principal est un simple document \LaTeX,
utilisant le style épuré de Tufte-\LaTeX{}\footnote{\url{GitHub.com/Tufte-LaTeX/tufte-latex}}.
%
Les sources sont en accès libre\footnote{\url{frama.link/7Vzgqtgx}} et sous licence Creative Commons.

Ce recueil à été rédigé, imprimé et relié avec amour en janvier 2019.


\hfill{} -- \emph{Lilian Besson}.

\section*{Remarque importante}
\textbf{Aucun bénéfice financier n'a été ni ne sera tiré de ce document.}
C'est juste pour s'amuser !
