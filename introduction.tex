\chapter{Introduction}

\vspace*{-30pt}

Chères amies et chers amis, merci d'être ici aujourd'hui\footnote{Sorry for the non native French speaker, this document is in French. You are in a team with at least one French speaker, ask him or ask me if you need more details.} !
%
Vous voilà réunis par \textbf{équipes de \intervalparequipe{} personnes},
et vous avez \textbf{\nbenigmes{} tâches à effectuer}.
%
Vous avez jusqu'à la fermeture du musée, à \textbf{17h50} !
%
Rendez-vous à la sortie, \textbf{à 18h.}


\section*{Consignes}

L'ordre des énigmes est \emph{aléatoire}, elles ne sont triées ni par ordre chronologique, ni logique, ni spatial dans le musée, et n'ont aucune dépendance entres elles.
%
Toutes peuvent être résolues sans enfreindre le règlement intérieur du musée.
\textbf{Pas de triche} : n'utilisez pas vos téléphones intelligents pour chercher sur Internet !

Munissez vous d'un appareil photo ou de vos \emph{téléphones intelligents} : \textbf{toutes les énigmes demandent de trouver une œuvre et de la prendre en photo}\footnote{Sans flash ! Pas de retouchage des photos, non plus !}.
Les énigmes ne concernent que la collection du premier étage, mais bien sûr vous pouvez visiter le rez-de-chaussée si vous le souhaitez !
Économisez votre batterie et relayez vous.
%
Vous êtes réunis dans une équipe, pensez donc à mobilisez les idées de tout le monde (et liez connaissance si vous ne vous connaissez pas) !

L'objectif n'est pas d'être l'équipe la plus rapide mais celle qui répond à toutes les énigmes !
La coopération entre les autres équipes n'est pas recommandée.
La meilleure équipe recevra un prix à la suite de la visite !
%
Un dernier conseil : restez calmes et discrets… il ne faut pas que les vigiles détectent que vous vous êtes lancés dans une chasse aux trésors…


\section*{Bonne chance !}
Affûtez votre regard, aiguisez votre attention, entrez dans le Musée des Beaux Arts, et vous voilà près à affronter (pacifiquement) les autres équipes !
